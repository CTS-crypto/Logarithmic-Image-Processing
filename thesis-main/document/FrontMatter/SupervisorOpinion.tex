\begin{opinion}
    El procesamiento de imágenes digitales mediante estructuras algebraicas (modelos no lineales) es un tema interesante por su baja demanda de recursos computacionales y la alta calidad de las imágenes procesadas. Dichas estructuras tienen una fundamentación matemática muy robusta y una interpretación física desde la óptica.
    
    La tesis presentada por Carlos Toledo Silva propone un módulo de Python que reúne varios modelos no lineales para el procesamiento de imágenes, y permite realizar operaciones diversas dentro de estos. Además, se realiza un estudio detallado de las ventajas y desventajas de cada modelo para fusionar imágenes y detectar bordes. Durante el proceso de investigación Carlos estudió la materia referida, que no está incluida en el currículo de la carrera y trabajó mostrando creatividad, disciplina, entrega y rigor. Deseo destacar su profundización en el estudio numérico y computacional de los modelos considerados en la tesis. Además, mostró creatividad para proponer soluciones a problemas de implementación y competencias de programación en el lenguaje Python y sus diversos \textit{frameworks}
    
    Es importante destacar que esta tesis se ubica en el marco de un proyecto Métodos numéricos para problemas en múltiples escalas, asociado al Programa Nacional de Ciencias Básicas, Código PN223LH010-003, Ministerio de Ciencia, Tecnología y Medio Ambiente (CITMA), Cuba, 2021-2023. Por lo que sus resultados serán aplicados para el procesamiento de imágenes médicas próximamente, de ahí la importancia de contar con este módulo de software en nuestro grupo de investigación.
    
    A pesar de los tropiezos propios del proceso investigativo, Carlos supo sopreponerse y plantear soluciones a las cuestiones referidas al inicio de la tesis, por lo que considero se logró cumplir su objetivo.
    Por tanto, recomiendo que al estudiante Carlos Toledo Silva debe otorgársele la máxima calificación (5 puntos, Excelente), y estoy seguro que en el futuro se desempeñará como un excelente profesional de la Ciencia de la Computación.
\end{opinion}