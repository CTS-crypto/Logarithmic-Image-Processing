\begin{resumen}
	En este trabajo se hace un an\'alisis de los principales modelos no lineales para el procesamiento de im\'agenes propuestos hasta la fecha y sus propiedades. Se hace tambi\'en un recuento de los antecedentes matem\'aticos que dan sustento a estos modelos pasando por la definici\'on de las leyes, la propiedad de cierre, las definiciones de espacio vectorial y cono, etc. Adem\'as, se propone un nuevo modelo no lineal parametrizado al cual se le otorg\'o el nombre de Modelo Pseudo-logar\'itmico Parametrizado para el Procesamiento de Im\'agenes (PPSLIP), que es una modificaci\'on del Modelo Pseudo-logar\'itmico para el Procesamiento de Im\'agenes (PSLIP). Se presenta, adem\'as, un m\'odulo de Python con la implementaci\'on de todos estos modelos. En dicho m\'odulo tambi\'en aparecen implementados algoritmos para el procesamiento de im\'agenes que permiten utilizar estos modelos. Entre los algoritmos implementados se tiene uno que se utiliza para la detecci\'on de bordes usando un filtro especificado, una variaci\'on del algoritmo de \textit{unsharp masking} y otro algoritmo que realiza modificaciones en el histograma de una imagen simulando una ecualizaci\'on. Tambi\'en aparece la implementaci\'on de las m\'etricas utilizadas en este trabajo: la Medida de Mejora por Entropía (EMEE) que se utiliza para obtener los mejores par\'ametros para una determinada operaci\'on en el modelo PLIP y el Contraste Promedio de un P\'ixel en una Imagen ($C_p$) que se utiliz\'o para evaluar los resultados en los experimentos realizados. Se realiza una comparaci\'on entre estos modelos y el modelo lineal a trav\'es de una serie de experimentos, utilizando los diferentes algoritmos implementados, mostrando as\'i ventajas y desventajas de estos.
\end{resumen}

\begin{abstract}
	This paper analyzes the main nonlinear models for image processing proposed to date and their properties. There is also a recount of the mathematical background that supports these models through the definition of the laws, the closing property, the definitions of vector space and cone, etc. In addition, a new parameterized nonlinear model is proposed, which was given the name of Parameterized Pseudo-logarithmic Model for Image Processing (PPSLIP), which is a modification of the Pseudo-logarithmic Model for Image Processing (PSLIP). A Python module is also presented with the implementation of all these models. In this module are also implemented algorithms for image processing that allow using these models. Among the algorithms implemented are one that is used for edge detection using a specified filter, a variation of the unsharp masking algorithm and another algorithm that makes modifications to the histogram of an image simulating an equalization. The implementation of the metrics used in this work also appears: the Improvement Measure by Entropy (EMEE) that is used to obtain the best parameters for a given operation in the PLIP model and the Average Contrast of a Pixel in an Image ($C_p$) that was used to evaluate the results in the experiments performed. A comparison is made between these models and the linear model through a series of experiments, using the different algorithms implemented, thus showing advantages and disadvantages of these.
\end{abstract}