\begin{conclusions}
    En este trabajo se repasaron los antecedentes matem\'aticos que dan sustento a los modelos no lineales. Se hizo un repaso por algunos de los m\'as conocidos y se analizaron sus propiedades. M\'as adelante se present\'o un nuevo modelo no lineal parametrizado y se describi\'o el m\'odulo implementado; el cual contiene la implementaci\'on de todos los modelos vistos, as\'i como de algoritmos que pueden usar estos modelos. Tambi\'en se describieron las m\'etricas objetivas utilizadas y se explicaron las modificaciones realizadas a una de estas con tal de obtener resultados m\'as consecuentes. Adem\'as, se explicaron algunos detalles de implementaci\'on del m\'odulo implementado. Por \'ultimo, se realizaron experimentos donde se compararon el modelo lineal y todos los modelos no lineales abordados en este trabajo. Se presentaron ejemplos ilustrativos y estad\'isticas obtenidas a partir de la realizaci\'on de varios experimentos. Con la realizaci\'on de todo lo anteriormente mencionado se arribaron a las siguientes conclusiones:
    
    El modelo lineal muestra deficiencias en determinadas ocasiones, las cuales pueden ser resueltas por alguno de los modelos no lineales presentados. El modelo LIP presenta sensibilidad hacia las tonalidades oscuras, mientras que el SLIP sin expansi\'on presenta sensibilidad hacia las tonalidades claras. Esto debido a que en el primero la escala de grises se invierte y en el segundo no. El modelo HLIP demostr\'o ser muy efectivo en los diferentes experimentos realizados dado su car\'acter sim\'etrico. Ser\'ia interesante hacer un estudio para el modelo SLIP expandiendo las im\'agenes por todo el rango de valores de este modelo para aprovechar la simetr\'ia, pero eso ya es un tema para otro trabajo. El modelo PSLIP demostr\'o ser muy efectivo para la detecci\'on de bordes, fundamentalmente en los niveles de mayor intensidad, quedando a deber en los de menor intensidad. Una soluci\'on a esto ser\'ia invertir la escala de grises, como se hace en el modelo LIP, si solo se quiere una sensibilidad a tonalidades oscuras o hacerlo sim\'etrico si se desea la doble sensibilidad como en el caso del modelo HLIP. Este tambi\'en es otro estudio que podr\'ia realizarse. La parametrizaci\'on de los modelos demostr\'o tener gran utilidad ya que permite cierta flexibilidad en favor de obtener mejores resultados. En concreto, el modelo parametrizado propuesto demostr\'o ser el mejor para la detecci\'on de bordes y para el algoritmo \textit{unsharp masking} en las im\'agenes m\'edicas utilizadas gracias a la parametrizaci\'on.
    
    El algoritmo de transformaci\'on af\'in con los dos modelos utilizados (HLIP y SLIP), si bien no obtuvo mejores resultados que la ecualizaci\'on del histograma en cuanto al valor del contraste promedio, en determinadas ocasiones, puede dar como resultado una imagen de mejor contraste que la imagen original y m\'as natural que la imagen ecualizada, lo cual es algo positivo. La m\'etrica $C_p$ utilizada para evaluar los experimentos puede ser utilizada para la evaluaci\'on de distintos algoritmos para el procesamiento de im\'agenes ya que cuenta con dos ventajas fundamentales: no necesita de una imagen de referencia y se puede especificar la escala en la que se encuentra la imagen para obtener resultados consecuentes independientemente de la escala en la que se encuentre la imagen, es decir, si se tiene una misma imagen en dos escalas diferentes, el valor de $C_p$ ser\'a el mismo en ambas escalas.
\end{conclusions}
