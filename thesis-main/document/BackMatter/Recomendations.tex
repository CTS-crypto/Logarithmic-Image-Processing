\begin{recomendations}
    Se recomienda la utilizaci\'on de los modelos no lineales abordados en este trabajo en sustituci\'on del modelo lineal en los casos estudiados donde se evidencia un mejor resultado; obviamente utilizar el m\'as adecuado seg\'un sea la situaci\'on. Tambi\'en se recomienda continuar el estudio de estos modelos ya que en este trabajo solo se abordaron algunos de ellos y algunas utilidades de los mismos. Este tema es ampliamente abarcador por lo que se recomienda estudiar otros modelos no lineales para el procesamiento de im\'agenes y experimentar con diferentes algoritmos para el procesamiento de im\'agenes haciendo uso de estos modelos. Por ejemplo, uno de estos podr\'ia ser un algoritmo que siguiera la idea del CLAHE de calcular los histogramas de manera local, pero utilizando el algoritmo de transformaci\'on af\'in con los modelos sim\'etricos. Tambi\'en se podr\'ian implementar algoritmos que combinasen dos o m\'as modelos aprovechando las ventajas que ofrece cada uno.
    
    Adem\'as, se recomienda que en los futuros estudios que se realicen, tambi\'en se incluya, como medida subjetiva, la opini\'on de diferentes usuarios con respecto a la calidad de los resultados obtenidos utilizando los diferentes modelos y algoritmos, algo que por cuestiones de tiempo no se pudo hacer en este trabajo.
    
    Otra recomendaci\'on que se propone es seguir profundizando en la implementaci\'on de un algoritmo, no necesariamente de Inteligencia Artificial, que permita estimar para un modelo parametrizado el mejor o los mejores par\'ametros para la realizaci\'on de una determinada operaci\'on. La concreci\'on de esta idea ser\'ia de gran utilidad para el desarrollo de algoritmos para el procesamiento de im\'agenes.
    
    
\end{recomendations}
