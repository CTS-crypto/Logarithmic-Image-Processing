\chapter{Estado del Arte}\label{chapter:state-of-the-art}

En este cap\'itulo se abordan, primeramente, los antecedentes matem\'aticos que dan soporte a los modelos no lineales. Luego, se analizan en profundidad algunos de los m\'as conocidos.

\section{Antecedentes Matemáticos}

\subsection{Definición de las Leyes}

Dado un escalar, $\alpha \in K \subseteq \mathbb{R}$, y dos elementos del conjunto $E$, $u$ y $v$, se pueden determinar las fórmulas exactas para las operaciones mencionadas utilizando la función generativa $\varphi$:
\begin{equation}
	\varphi(u\oplus v) = \varphi(u)+\varphi(v), \forall u,v \in E,
\end{equation}
\begin{equation}
	\varphi(\alpha \otimes u) = \alpha \cdot \varphi(u), \forall u \in E, \forall \alpha \in K.
\end{equation}

Las ecuaciones (1.1) y (1.2) son las condiciones que debe cumplir un homomorfismo entre dos estructuras algebraicas similares. La admisi\'on de un inverso y la existencia de biyectividad indica la presencia de un isomorfismo.

Ahora se analiza la ecuación (1.2). Mientras que $u$ puede ser cualquier elemento del conjunto de definici\'on $E$, $\alpha$ es, típicamente, un escalar real positivo $(K=\mathbb{R}^+)$.

Con respecto a la restricción de biyectividad y la existencia de $\varphi^{-1}$, las leyes de definición están determinadas por:
\begin{equation}
	u \oplus v = \varphi^{-1}(\varphi(u)+\varphi(v)),	
\end{equation}
\begin{equation}	
	\alpha \otimes u = \varphi^{-1}(\alpha\cdot\varphi(u)).
\end{equation} 

\subsection{Propiedad de Cierre}

La propiedad de cierre, tanto de la suma como de la multiplicación por un escalar, es de suma importancia práctica, ya que la suma de dos imágenes cualquiera debería conducir a otra imagen válida y, respectivamente, cualquier imagen amplificada o atenuada debería ser una imagen. Formalmente, se puede escribir:
\begin{equation}
	\forall u,v \in E, z = u \oplus v \Rightarrow z \in E,
\end{equation}
\begin{equation}
	\forall u \in E, \forall \alpha \in K, z=\alpha \otimes v \Rightarrow z \in E.
\end{equation}

Estas propiedades se mantienen bajo la hipótesis de biyectividad asumida ya que: 
\begin{equation}
	z=\varphi^{-1}(\varphi(u)+\varphi(v)) \land \forall x \in F, \varphi^{-1}(x) \in E \Rightarrow z \in E.
\end{equation}
\begin{equation}
	z=\varphi^{-1}(\alpha\cdot\varphi(u)) \land \forall x \in F, \varphi^{-1}(x) \in E \Rightarrow z \in E.
\end{equation}

\subsection{Espacio Vectorial y Cono}
Dadas las dos leyes operativas, $\oplus$, $\otimes$, el conjunto vectorial $E$ y el conjunto escalar $K$, la definición formal del espacio vectorial implica varias propiedades. La ley de la suma debe ser asociativa, conmutativa, debe tener elemento identidad y elemento inverso. La distributividad debe mantenerse para la multiplicación escalar sobre la suma de vectores y para la multiplicación escalar en el espacio de los escalares. La multiplicación escalar debe tener elemento identidad y ser compatible con la multiplicación en el espacio de los escalares.

Las propiedades conmutativas, asociativas y distributivas de las leyes implícitas son importantes porque el orden de las operaciones no debería importar en una suma ponderada de imágenes. Bajo la hipótesis asumida de aplicación biyectiva y debido a que la función $\varphi^{-1}$ asigna la estructura real al conjunto dado, estas propiedades se verifican.

La existencia del elemento identidad, $u_0$, con respecto a la suma, implica condiciones adicionales sobre la función de mapeo, $\varphi$. La condición es una consecuencia del comportamiento isomorfo:
\begin{equation}
	\forall u\in E, \exists u_0 : u \oplus u_0 = u | \varphi(u_0)=0.
\end{equation}

El elemento inverso, $u^-$, para espacios asimétricos tiene una aplicación bastante poco práctica: ``\textquestiondown¡dado un valor de intensidad, el elemento inverso es algo que absorbe perfectamente la luz!?''. Este es más importante para definir la sustracción de una imagen de otra para que sea consistente con la suma. Sin embargo, el elemento inverso tiene sentido si se habla de espacios de color simétricos. En tal caso, la función de mapeo debe tomar valores en un intervalo simétrico:
\begin{equation}
	\forall u \in E, \exists u^- \Rightarrow u \oplus u^- = u_0 | \varphi(u) + \varphi(u^-) = 0 \Rightarrow \varphi(u^-)=-\varphi(u). 
\end{equation}

De manera similar, el elemento identidad de la multiplicación escalar tiene que ser 1:
\begin{equation}
	\forall u \in E - \{u_0\}, \exists \alpha_1 : \alpha_1 \otimes u = u | \alpha_1 = 1.
\end{equation}

Otro concepto importante es el de cono. Siendo $V$ un espacio vectorial real y $K$, el conjunto escalar. Un cono $C$ en $V$ es un subconjunto de $V$ que satisface~\cite{barker1981theory}:
\begin{enumerate}
	\item si $u_1,u_2\in C$ y $\alpha,\beta\in K:\alpha,\beta\geq0$, entonces $\alpha u_1+\beta u_2\in C$.
	\item $C \cap (-C)={0}$.
\end{enumerate}

\section{Modelo LIP}

El Modelo Logar\'itmico Cl\'asico para el Procesamiento de Im\'agenes (LIP) fue el primero de estos modelos en ser presentado. Fue introducido por Michel Jourlin y Jean-Charles Pinoli~\cite{jourlin1988model}. En el modelo LIP, una imagen se representa mediante su función de tono de gris asociada, denominada $f$, que se define en el dominio espacial no vacío $D \subset \mathbb{R}^2$. Una función de tono de gris toma valores en el intervalo acotado de números reales $[0, M)$, donde $M$ es estrictamente positivo, denominado rango de tonos de gris. Los elementos de $[0, M )$ se denominan tonos de gris. El valor de un tono de gris $v$ est\'a asociado a su valor de nivel de gris $u$ correspondiente de la siguiente manera:
\begin{equation}
	v = M - u.
\end{equation}

La escala de intensidad se invierte de manera que 0 representa la blancura o transparencia total y $M$ representa la negrura u opacidad absoluta. En el modelo LIP, la operación de adición se define para sumar dos imágenes de luz transmitida. Físicamente, la adición de dos imágenes de luz transmitida sigue la ley de transmitancia clásica y no se puede alcanzar la negrura total. Como tal, la inversión de escala está justificada, porque el 0 corresponde a la transparencia total y es, en esta caso, el elemento neutral para una suma matemática.

El objetivo inicial del modelo LIP era definir una operación aditiva cerrada en el rango acotado de intensidad de números reales positivos $[0, M)$~\cite{jourlin1988model}, es decir, un cono positivo $[0, M )$. La suma de dos tonos de gris $v_1$ y $v_2$ se define como:
\begin{equation}
	v_1\oplus v_2=v_1+v_2-\frac{v_1v_2}{M}
\end{equation}
y la multiplicación por un escalar real positivo $\lambda \in \mathbb{R}^+$ se define como:
\begin{equation}
	\lambda \otimes v = M - M\left(1-\frac{v}{M}\right)^\lambda
\end{equation}

La suma $\oplus$ es asociativa, conmutativa y tiene un elemento identidad: 0. La multiplicaci\'on escalar $\otimes$ es asociativa, tiene elemento identidad: 1 y la distributividad se mantiene para la multiplicación escalar sobre la suma $\oplus$ y para la multiplicación escalar en el espacio de los escalares~\cite{jourlin1988model}~\cite{jourlin2016logarithmic}.

En orden de extender el cono positivo a un espacio vectorial, el rango de tonos de gris es extendido de $[0,M)$ a $(-\infty,M)$~\cite{jourlin1988model}~\cite{jourlin2016logarithmic}. Esto permiti\'o entonces, extender la multiplicaci\'on por un escalar a cualquier n\'umero real $\lambda \in \mathbb{R}$, permitiendo a su vez definir el negativo y la resta como:
\begin{equation}
	\ominus v=-\frac{v}{1-\frac{v}{M}},
\end{equation}
\begin{equation}
	v_1 \ominus v_2 = \frac{v_1-v_2}{1-\frac{v_2}{M}}.
\end{equation}

El conjunto tonos de gris en el rango $(-\infty, M)$ está relacionado con $\mathbb{R}$ a través del isomorfismo fundamental~\cite{jourlin2016logarithmic} definido como:
\begin{equation}
	\varphi(v) = -M\ln\left(1-\frac{v}{M}\right),
\end{equation}
y su inverso:
\begin{equation}
	\varphi^{-1} (x) = M\left(1-e^{-\frac{x}{M}}\right).
\end{equation}

Obs\'ervese que sustituyendo (1.11) en (1.16) para un nivel de gris $u$ se puede obtener su correspondiente en el isomorfismo mediante la funci\'on:
\begin{equation}
	\phi(u) = -M\ln\left(1-\frac{M-u}{M}\right) = -M\ln\left(\frac{u}{M}\right)
\end{equation}
y su inversa se define como:
\begin{equation}
	\phi^{-1}(x) = Me^{-\frac{x}{M}} 
\end{equation}

Las operaciones para el conjunto de im\'agenes $I(D,E)$ se extienden, aplicando la operaci\'on en cuesti\'on a cada p\'ixel de la imagen (funci\'on de cambio a tono de gris, multiplicaci\'on escalar, funci\'on de isomorfismo, etc.) o parejas de p\'ixeles que se encuentran en la misma posici\'on en dos im\'agenes diferentes (suma, resta, etc). 

\section{Modelo HLIP}

El Modelo Logar\'itmico Homom\'orfico para el Procesamiento de Im\'agenes (HLIP) fue presentado por Vasile Pătraşcu y Vasile Buzuloiu~\cite{patrascu2014mathematical}. Dentro del modelo matemático desarrollado, el conjunto de tonos de gris es el intervalo acotado $E = (-1, 1)$ puesto en correspondencia uno a uno con el intervalo de niveles de gris $(0,M)$, por la transformaci\'on lineal:
\begin{equation}
	v=\frac{2}{M}\left(u-\frac{M}{2}\right).
\end{equation}

En el conjunto de tonos de gris $E$, se define la suma $\oplus$ y la multiplicación escalar real $\otimes$.

$\forall v_1,v_2 \in E$ la suma es definida por la siguiente relaci\'on:
\begin{equation}
	v_1 \oplus v_2=\frac{v_1+v_2}{1+v_1v_2},
\end{equation}
donde las operaciones en el lado derecho se mantienen en $E$.

El elemento neutro para la suma es $\theta = 0$. Cada elemento $v \in E$ tiene como opuesto al elemento $w = - v$ y esto verifica la siguiente ecuación: $v \oplus w = \theta$.

La suma $\oplus$ es asociativa, conmutativa, tiene un elemento neutro y cada elemento tiene un opuesto. Significa que esta operación establece en $E$ una estructura de grupo conmutativo.

La resta puede ser tambi\'en definida como:
\begin{equation}
	v_1\ominus v_2=\frac{v_1-v_2}{1-v_1v_2}.
\end{equation}

Adem\'as $\forall \lambda \in \mathbb{R}, \forall v \in E $, se define el producto entre $\lambda$ y $v$ por:
\begin{equation}
	\lambda \otimes v =\frac{(1+v)^\lambda-(1-v)^\lambda}{(1+v)^\lambda+(1-v)^\lambda},
\end{equation}
donde nuevamente las operaciones en el lado derecho de la igualdad se mantienen en $E$. Las dos operaciones, suma $\oplus$ y multiplicación escalar $\otimes$ establecen una estructura de espacio vectorial.

El espacio vectorial de tonos de gris $(E, \mathbb{R}, \oplus, \otimes)$ es isomorfo al espacio de números reales $(\mathbb{R},\mathbb{R}, +, \cdot)$ por la función $\varphi(v) : E \rightarrow \mathbb{R}$, definida como:
\begin{equation}
	\varphi(v)=\frac{1}{2}\ln\left(\frac{1+v}{1-v}\right),
\end{equation}
y su inversa es la funci\'on:
\begin{equation}
	\varphi^{-1}(x)=\frac{e^{2x}-1}{e^{2x}+1}.
\end{equation}

Las operaciones para el conjunto de im\'agenes $I(D,E)$ se extienden como se explic\'o en el modelo anterior.

\section{Modelo PSLIP}

El Modelo Pseudo-logar\'itmico para el Procesamiento de Im\'agenes (PSLIP) fue propuesto por Costantin Vertan y otros colaboradores~\cite{vertan2008pseudo}. Su aplicaci\'on principal es para la detecci\'on de bordes. En este modelo para alg\'un nivel de gris $u\in[0,M)$, la relaci\'on con su tono de gris $v$ correspondiente viene dada por:
\begin{equation}
	v = \frac{u}{M},
\end{equation}
de donde $v\in[0,1)$.

Se define la suma de dos tonos de gris $v_1,v_2\in[0,1)$ como:
\begin{equation}
	v_1\oplus v_2=\frac{v_1 + v_2 - 2v_1v_2}{1-v_1v_2},
\end{equation}
mientras que la multiplicación de un tono de gris, $v$ con un escalar real positivo, $\lambda \in \mathbb{R}^+$ es:
\begin{equation}
	\lambda \otimes v = \frac{\lambda v}{1+(\lambda-1)v}.
\end{equation}

La diferencia entre dos tonos de gris $v_1$ y $v_2$, con $v_1 \geq v_2$ viene dada por:
\begin{equation}
	v_1\ominus v_2=\frac{v_1-v_2}{1+v_1v_2-2v_2}.
\end{equation}

Se puede demostrar que el intervalo $[0, 1)$ es isomorfo con $\mathbb{R}^+$ a través de la transformaci\'on $\varphi$ definida como:
\begin{equation}
	\varphi(v)=\frac{v}{1-v},
\end{equation}
y su inversa:
\begin{equation}
	\varphi^{-1}(x)=\frac{x}{1+x}.
\end{equation}

Las operaciones para el conjunto de im\'agenes $I(D,E)$ se extienden como se explic\'o en los modelos anteriores.

Teniendo en cuenta que la función generadora es una biyección, con valores en $F = [ 0 , +\infty )$ y $\varphi(0)=0$, el modelo cumple con todas las propiedades de una estructura de espacio cónico. La extensión a una estructura de espacio vectorial se puede lograr mediante el uso de la función generadora $\varphi:(-1,1)\to (-\infty,+\infty)$~\cite{florea2009piecewise}:
\begin{equation}
	\varphi(v)=\frac{v}{1-|v|}.
\end{equation}

\section{Modelo PLIP}

El Modelo Logar\'itmico Parametrizado para el Procesamiento de Im\'agenes (PLIP) fue propuesto por Karen Panetta y otros colaboradores~\cite{panetta2007parameterization}~\cite{panetta2010parameterized}. Este incluye los modelos LIP y lineal como instancias especiales y también abarca todos los casos intermedios en un solo modelo unificado.

Las operaciones aritméticas hacen uso de una función de tono de gris parametrizada. Las operaciones PLIP se muestran en la Tabla 1.1.

\begin{table}[h]
	\caption{Resumen de las operaciones aritm\'eticas LIP y PLIP.}
	\begin{center}
		\begin{tabular}{|l|l|}
			\hline 
			\textbf{LIP} & \textbf{PLIP}\\
			\hline
			$v=M-u$ & $v=\mu(M)-u$\\
			\hline
			$v_1\boxplus v_2=v_1+v_2-\frac{v_1v_2}{M}$ & $v_1\oplus v_2=v_1+v_2-\frac{v_1v_2}{\gamma(M)}$\\
			\hline
			$v_1\boxminus v_2=\frac{v_1-v_2}{1-\frac{v_2}{M}}$ & $v_1\ominus v_2=\frac{v_1-v_2}{1-\frac{v_2}{k(M)}}$\\
			\hline
			$c\boxtimes v=M-M(1-\frac{v}{M})^c$ & $c\otimes v=\gamma(M)-\gamma(M)(1-\frac{v}{\gamma(M)})^c$\\
			\hline
			$\varPhi(v)=-M\ln(1-\frac{v}{M})$ & $\varphi(v)=-\lambda(M)\ln^\beta(1-\frac{v}{\lambda(M)})$\\
			\hline
			$\varPhi^{-1}(x)=M\left(1-e^{-\frac{x}{M}}\right)$ & $\varphi^{-1}(x)=\lambda(M)\left(1-\left(e^{-\frac{x}{\lambda(M)}}\right)^{\frac{1}{\beta}}\right)$\\
			\hline
		\end{tabular}
	\end{center}
\end{table}

El valor de $\mu(M)$ utilizado para calcular la función de tono de gris podría depender de la imagen, como el valor máximo de la imagen, o podría ser un valor mayor, como $\mu(M) = 1026$. También se observa que la suma y la multiplicación escalar usan la misma función $\lambda(M)$. Esto se debe a que la multiplicación escalar es una extensión de la suma, sumando la imagen a sí misma $c$ veces ($c>0$).

Además, se propone un coeficiente exponencial $\beta>0$ a la función fundamental del isomorfismo $\varphi$. Al agregar este coeficiente exponencial $\beta$, es posible ajustar la sensibilidad al extremo oscuro del rango de intensidad de píxeles ($\beta$ pequeño) o al extremo brillante del rango de intensidad de píxeles de $[0, M)$ ($\beta$ grande).

Propiedades del modelo PLIP~\cite{panetta2010parameterized}:

\begin{enumerate}
	\item En el modelo PLIP, las operaciones aritméticas lineales se reemplazan con nuevas operaciones de la misma manera que en el modelo LIP tradicional.
	\item Las operaciones en el modelo PLIP son iguales a las del modelo tradicional LIP cuando $\mu(M) = \gamma(M) = k(M) = \lambda(M) = M$ y $\beta = 1$.
	\item Las operaciones en el modelo PLIP se asemejan a las operaciones aritméticas lineales a medida que $\gamma(M),~k(M)$ y $\lambda(M)$ se aproximan al infinito y $\beta = 1$
	\item Las operaciones PLIP pueden generar más casos entre los dos casos extremos del LIP y las operaciones aritméticas lineales cuando los parámetros $\mu, \gamma, k$ y $\lambda$ cambian dentro de $[M, +\infty)$, como se muestra en la Figura 1.1.
	\item Se puede demostrar que las operaciones PLIP cumplen las leyes de asociatividad, conmutatividad, elemento identidad y las propiedades distributivas.
\end{enumerate}

\begin{figure}
	\begin{center}
		\includegraphics[width=8.0 cm]{images/plip_scheme.png}
		\caption{Operaciones en el modelo PLIP}
	\end{center}
\end{figure}

El objetivo de la parametrizaci\'on es disminuir la p\'erdida de informaci\'on. En esencia, cuando se agrega una imagen visualmente ``buena'' a otra imagen visualmente ``buena'', el resultado también debe ser ``bueno''. Esto es de particular importancia, por ejemplo, cuando se recibe información de dos sensores que deben fusionarse de alguna manera.

\subsection{EMEE}

Para designar una imagen como visualmente ``buena'', es necesario establecer unos criterios objetivos de cuantificación de los resultados. La mejora automática de imágenes basada en los requisitos visuales humanos sigue siendo un problema muy desafiante. No existe un método único de mejora de imágenes que funcione bien para todas las imágenes. Este problema se vuelve más apremiante cuando se necesita mejorar miles de imágenes en un entorno automatizado.

La Medida de Mejora por Entropía (EMEE, en ingl\'es) ha demostrado ser una medida eficaz para evaluar la calidad de mejora de la imagen~\cite{agaian2000new}. Esta medida se basa en las mismas leyes psicovisuales que forman la base de los modelos LIP y PLIP. La EMEE se calcula dividiendo una imagen $I$ en $k_1 \times k_2$ bloques, obteniendo el máximo local $I_{max k,l}$ y el mínimo $I_{min k,l}$ dentro de cada bloque individualmente, y luego procesándolos usando la siguiente ecuación:
\begin{equation}
	\displaystyle EMEE_{\alpha,k_1,k_2=\frac{1}{k_1k_2}}\sum_{l=1}^{k_1}\sum_{k=1}^{k_2}\alpha\left(\frac{I_{max k,l}}{I_{min k,l}}\right)^\alpha\ln\left(\frac{I_{max k,l}}{I_{min k,l}}\right),
\end{equation}
donde $\alpha$ es una constante que puede ayudar a seleccionar los parámetros. Se eligi\'o $\alpha = 1$ y el tamaño de bloque $4 \times 4$, $4 \times 5$, $5 \times 4$ y $5 \times 5$, seg\'un las dimensiones de la imagen, para calcular los resultados de EMEE en esta tesis.

El mejor parámetro (óptimo) se obtiene si se cumple la siguiente condición:
\begin{equation}
	EMEE_{optimal}=m\'ax_{local}(EMEE(\alpha, \mu, \gamma, k, \lambda, \beta))
\end{equation}
donde $EMEE_{optimal}$ es el valor $EMEE$ optimizado para la imagen, $m\'ax_{local}(X)$ es una función para obtener el valor máximo local de $X$ y $EMEE(\alpha, \mu, \gamma, k, \lambda, \beta)$ es la medida $EMEE$ resultante cuando cambian los parámetros $\alpha$, $\mu$, $\gamma$, $k$, $\lambda$ y $\beta$.

\section{Modelo SLIP}

Las operaciones LIP están acotadas en la parte positiva $[0, M )$ e ilimitadas en la parte negativa. Por ejemplo, el resultado de la siguiente operación de resta podría ser ilimitado cuando $v_1 < v_2$:
\begin{equation}
	v_1 \ominus v_2 \in (-\infty,0).
\end{equation}

Esta propiedad conduce potencialmente al problema de fuera de rango. Para solucionar dicho problema un nuevo modelo sim\'etrico es presentado por Laurent Navarro y otros colaboradores~\cite{navarro2013symmetric}: el Modelo Logar\'itmico Sim\'etrico para el Procesamiento de Im\'agenes (SLIP). 

En el modelo SLIP, una imagen está representada por su función de nivel de gris asociada, denominada $f$, definida en el dominio espacial no vacío $D \subset \mathbb{R}^2$. Las funciones de nivel de gris toman valores en el intervalo simétrico acotado de números reales $(-M, M)$, donde $M$ es estrictamente positivo, denominado rango de niveles de gris. Los elementos de $(-M, M )$ se denominan niveles de gris. $M$ representa la intensidad de luz máxima y $-M$ es la absorción de luz total.

El modelo SLIP se basa en un isomorfismo impar inspirado en el isomorfismo del modelo LIP para obtener un modelo que tenga el mismo comportamiento para valores positivos y negativos. En el modelo SLIP, el isomorfismo fundamental y su inversa se definen como:
\begin{equation}
	\varphi(v)=-Msgn(v)\ln\left(1-\frac{|v|}{M}\right),
\end{equation}
y
\begin{equation}
	\varphi^{-1}(x)=Msgn(x)\left(1-e^{-\frac{|x|}{M}}\right).
\end{equation}

La adición de dos niveles de gris $v_1$ y $v_2$ se define como:
\begin{equation}
	v_1\oplus v_2=Msgn(v_1+v_2)\left[1-\left(1-\frac{|v_1|}{M}\right)^{\gamma_1}\left(1-\frac{|v_2|}{M}\right)^{\gamma_2}\right],
\end{equation}
donde
\begin{equation}
	\gamma_1=\frac{sgn(v_1)}{sgn(v_1+v_2)},
\end{equation}
y
\begin{equation}
	\gamma_2=\frac{sgn(v_2)}{sgn(v_1+v_2)}.
\end{equation}

La multiplicación por un escalar $\lambda (\lambda \in \mathbb{R})$ se define como:
\begin{equation}
	\lambda \otimes v = Msgn(\lambda v)\left[1-\left(1-\frac{|v|}{M}\right)^{|\lambda|}\right].
\end{equation}

Dado que el modelo SLIP se define desde el punto de vista del espacio vectorial, su operación de suma $\otimes$ satisface los siguientes axiomas de grupos abelianos~\cite{navarro2013symmetric}:

\begin{enumerate}
	\item El espacio se cierra bajo la operación, es decir, $v_1\oplus v_2 \in (-M, M )~\forall v_1, v_2 \in (-M, M )$.
	\item La operación es asociativa.
	\item La operación es conmutativa.
	\item El elemento identidad es $0$ tal que $0\oplus v = v$.
	\item  Existe un opuesto $w, \forall v \in (-M, M ): v \oplus w = 0$.
\end{enumerate}
  
Además, el conjunto de señales se cierra bajo la operación de multiplicación escalar, es decir, $\lambda \otimes v \in (-M, M), \forall \lambda \in \mathbb{R}, \forall v \in (-M, M)$. La operación también es distributiva en que $\lambda \otimes (v_1\oplus v_2) = (\lambda \otimes v_1 )\oplus (\lambda\otimes v_2)$ y $\beta\otimes (\lambda\otimes v_1 ) = (\beta \cdot \lambda)\otimes v_1, \forall \beta, \lambda \in \mathbb{R} \land v_1 , v_2 \in (-M, M)$. Su elemento de identidad es 1, es decir, $1\otimes v = v $.

Por lo tanto, el conjunto es cerrado para estas dos operaciones. Estas propiedades son esenciales para resolver el problema del fuera de rango. Además, es fácil demostrar que el valor opuesto $w$ se define como $w = (-1)\otimes v$ y la operación de resta se puede definir como: $v_1\ominus v_2=v_1\oplus(-1)\otimes v_2$.

\subsection{Relación Matemática entre los Modelos LIP y SLIP}

Desde un punto de vista matemático, se puede ver claramente que cuando $v \in [0, M)$, las dos funciones generadoras $\varphi_{LIP}(v)$ y $\varphi_{SLIP}(v)$ son exactamente las mismas (sin tener en cuenta la inversi\'on de la escala de grises). Por lo tanto, el modelo SLIP($\odot$) es el mismo que el modelo LIP($\boxdot$) para el rango de valores de píxeles $[0, M)$. Sin embargo, difieren entre sí en la forma en que tratan los valores negativos.  Por ejemplo, cuando $\lambda < 0$ y $v > 0$, tenemos $\lambda\boxtimes v \in (-\infty, 0)$ y $\lambda \otimes v \in (-M, 0)$. Del mismo modo, cuando $-\infty < v_1 < v_2 < M$ se tiene $v_1 \boxminus v_2 \in (-\infty, 0)$ y cuando $-M < v_1 < v_2 < M$, se tiene $v_1\ominus v_2 \in (-M, 0)$. 
