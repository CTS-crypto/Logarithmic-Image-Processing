\chapter*{Introducción}\label{chapter:introduction}
\addcontentsline{toc}{chapter}{Introducción}

En la mayoría de las circunstancias imaginables, las imágenes digitales se obtienen por medios que implican sensores con fuente de alimentación finita; por lo tanto, las imágenes digitales se definen en un rango finito de valores. Los algoritmos para el procesamiento de imágenes, tradicionalmente, se basan en operaciones reales clásicas para su implementación. Bajo ciertas circunstancias, tal combinación, denominada Procesamiento Lineal Clásico de Imágenes (CLIP) demuestra sus limitaciones. Por ejemplo, se puede mencionar el desbordamiento del rango superior, que es cuando el nivel de intensidad de un p\'ixel excede el m\'aximo admisible y se resuelve mediante el truncamiento. En consecuencia, aparecen estructuras más elaboradas, como los modelos para el Procesamiento Logarítmico de Imágenes (LIP).

El punto de partida de los modelos logarítmicos para el procesamiento de imágenes radica en la teoría homomórfica introducida por Alan V. Oppenheim~\cite{oppenheim1965superposition}. Las primeras implementaciones de los modelos LIP han sido proporcionadas por Michel Jourlin y Jean-Charles Pinoli~\cite{jourlin1988model}, as\'i como por Vasile Pătraşcu y Vasile Buzuloiu~\cite{patrascu2014mathematical}. Luego el esquema de un nuevo modelo pseudo-logarítmico fue propuesto por Constantin Vertan y otros colaboradores~\cite{vertan2008pseudo}. Los otros modelos que se abordan en este trabajo est\'an inspirados en los anteriores. Con estos modelos se han desarrollado diversas aplicaciones: corrección de iluminación~\cite{patrascu2001modele}, mejora de contraste~\cite{deng1995study}, mejora de imagen en color~\cite{patrascu2001modele}, ecualización de histogramas~\cite{puatracscu2003fuzzy}, mejora de rango dinámico~\cite{florea2008use}, detección de bordes~\cite{vertan2008pseudo}, etc.

La primera versi\'on de estos modelos, propuesta por Jourlin y Pinoli~\cite{jourlin1988model}, se ha desarrollado para el caso de la luz transmitida. La construcción matemática comienza definiendo la suma de dos imágenes $f$ y $g$, interpretable como la superposición de los obstáculos (objetos) generando, respectivamente, $f$ y $g$; la multiplicación se deriva por inducción de la suma repetida; las propiedades consiguientes surgen naturalmente. A diferencia de sus predecesores, Pătraşcu y Buzuloiu~\cite{patrascu2014mathematical} derivaron su modelo desde un punto de vista matemático al imponer algunas propiedades definidas a las leyes básicas (suma y multiplicación por escalar).

La construcción matemática de un modelo no lineal puede comenzar definiendo las leyes operativas (la suma y la multiplicación escalar) o, de manera equivalente, mediante la determinación de una función que mapee el conjunto de definición del modelo investigado en la estructura algebraica de números reales.

Para tener un uso práctico, es de sentido común imponer algunas propiedades a cualquier modelo recién determinado. Para ser más precisos, se debe tener en cuenta la naturaleza del conjunto de definición, los medios de determinación de las leyes, las propiedades de cierre y los requisitos para formar un espacio vectorial. 

Consid\'erese una función, $\varphi : E \rightarrow F$. Donde el conjunto $E$ es el conjunto de definición de la imagen. Normalmente, si los valores de la imagen son intensidades, como cualquier plano en representación de color RGB, el conjunto $E$ tiene la forma $[0,M)$; en el caso del espacio YUV (YCbCr), para los canales de diferencias de color, el conjunto $E$ tiene una forma simétrica, como $( - M/2 , + M/2 )$. Así, en cualquier circunstancia, el conjunto está acotado: $\exists m_E=inf(E),\exists M_E=sup(E)$.

Una imagen en niveles de gris $f$ se define sobre un dominio $D$ incluido en el espacio $\mathbb{R}^2$ (o en el espacio $\mathbb{R}^3$  para imágenes 3D) y toma sus valores (niveles de gris) en la escala de grises $[0, M)$. Para imágenes de 8 bits, $ M = 256$ y los 256 niveles de gris están en la escala de números enteros $[0,\cdots, 255]$. Clásicamente, el 0 corresponde al extremo negro, pero es posible hacer la otra elección, siendo el 0 asociado al extremo blanco. El conjunto de imágenes de niveles o tonos de gris definidas en $D$ y que toman valores de $E$ se definir\'a como $I(D,E)$.

La función $\varphi$ define la estructura del modelo y asigna el conjunto de definición de imagen, $E$, a un subconjunto de números reales, $F$. El conjunto de estructuras vectoriales definidas en $D$ y que toman valores en $F$ se define como $I(D,F)$. O sea, una estructura pertenece a este conjunto si existe una imagen en $I(D,E)$ que al aplicarle la funci\'on $\varphi$ genera dicha estructura. Además, se suman dos operaciones al conjunto dado, $E$: suma de dos elementos del conjunto, $\oplus$ , y multiplicación por un escalar, $\otimes$.  

Dado lo planteado anteriormente ser\'ia de gran utilidad contar con un \textit{framework} para la realizaci\'on de las operaciones b\'asicas utilizando los modelos abordados, as\'i como la comparaci\'on de los mismos en diferentes situaciones para poder conocer y usar, seg\'un la situaci\'on, el m\'as adecuado para obtener los mejores resultados.

\section*{Objetivos}

Por lo planteado anteriormente uno de los objetivos de esta tesis, es el dise\~no e implementaci\'on de un m\'odulo de Python donde se encuentren los modelos no lineales para el procesamiento de im\'agenes abordados en este trabajo. En dicho m\'odulo aparecer\'an implementadas funcionalidades para transformar una imagen $f$ a su estructura vectorial correspondiente en el espacio deseado, lo cual permitir\'a realizar en dicho espacio las operaciones que se deseen. Tambi\'en aparecer\'a, por supuesto, para cada espacio, una funcionalidad que dada una estructura permita obtener su equivalente en el espacio original. En resumen, para cada modelo se implementar\'an las funciones $\varphi$ y $\varphi^{-1}$, as\'i como la definici\'on de los operadores de las diferentes operaciones b\'asicas (suma, resta, multiplicaci\'on y potenciaci\'on).

Otro objetivo es implementar, para cada uno de los modelos, las operaciones b\'asicas sin tener que pasar las im\'agenes a otro espacio. Es decir, seg\'un sea la operaci\'on, obtener resultados similares a si se hubieran transformado las im\'agenes al espacio en cuesti\'on, operado con dichas transformaciones y luego el resultado obtenido de dicha operaci\'on, transformarlo al espacio original. O sea, que se implementar\'an las operaciones: $\oplus$, $\ominus$, $\otimes$, otras propias del modelo (como las funciones de cambio a tono de gris y cambio a nivel de gris en algunos de los modelos) y otras \'utiles como la curva del isomorfismo.

Adem\'as, otro objetivo es la evaluaci\'on de los diferentes modelos en cuanto a las diferentes operaciones, algoritmos, tipos de im\'agenes y situaciones.

Para la realizaci\'on de cada uno de estos objetivos generales, se definen las siguientes tareas de investigaci\'on:

\begin{enumerate}
	\item Revisar la literatura referente a los modelos no lineales para el procesamiento de im\'agenes.
	\item Implementaci\'on en Python de los diferentes modelos a partir de la bibliograf\'ia consultada.
	\item Creaci\'on de \textit{datasets} con diferentes tipos de im\'agenes para evaluar los modelos.
	\item Selecci\'on de m\'etricas para la evaluaci\'on de los modelos.
	\item Evaluaci\'on de los modelos.
	\item Arribar a conclusiones sobre la eficacia y utilidad de estos modelos. 
\end{enumerate}

Esta tesis pertenece a acciones del proyecto ``Métodos numéricos para problemas en múltiples escalas''. Proyecto asociado al Programa Nacional de Ciencias Básicas, Código PN223LH010-003, Ministerio de Ciencia, Tecnología y Medio Ambiente (CITMA), Cuba, 2021-2023.

\section*{Organizaci\'on del resto del documento}

El resto del documento se organiza de la siguiente manera. En el Cap\'itulo 1 se hace un recuento de los antecedentes matem\'aticos que dan sustento a los modelos no lineales para el procesamiento de im\'agenes y se analizan los m\'as conocidos hasta la fecha, as\'i como sus propiedades. En el Cap\'itulo 2 se presenta un nuevo modelo no lineal PPSLIP y se discuten sus propiedades. Adem\'as, se hace una descripci\'on del m\'odulo implementado, al igual que de las m\'etricas y algoritmos codificados en el mismo. Tambi\'en se describe un algoritmo que se trat\'o de implementar para estimar los mejores par\'ametros en el modelo PLIP para una determinada operaci\'on. En el Cap\'itulo 3 se profundiza en los detalles de implementaci\'on del m\'odulo, se ilustran, explican y comentan los diferentes experimentos realizados en los que se comparan los diferentes modelos. Por \'ultimo, aparecen las conclusiones, recomendaciones y la bibliograf\'ia utilizada.